\section{Client User Guide}\label{Client User Guide} 
    
    \subsection{Intro}\label{userguideIntro}
    
    This is the user manual for the QoS client library. Here we will cover how to use the client library to communicate with our server implementation, in a way that properly set the diffserv value on the packages sent from the client to the server. This manual is intended to be as simplistic and understandable as possible.

    First we will cover the steps you need to take in order to be able to use our library. Then move on to the easiest way to use it, as well as how to use it with a listener pattern. This will be followed by a section covering the available settings in the library, and finally we will bring up some caveats of our implementation that is worth keeping in mind.

    \subsection{Required interfaces}\label{userguideReqiredInterfaces}
    
    One of the most important parts of all network communication is being able to catch and handle exceptions gracefully. In order for a client, the user of our library, to be notified of exceptions that occur they will have to implement the ExceptionHandler interface found in no.ntnu.qos.client.ExceptionHandler. This interface contains several methods, which are:

\lstset{language=java}
\lstset{frame=single}
\lstset{breaklines=true}
\begin{lstlisting}[caption={ExceptionHandler interface}, label=userguideExceptionHandler]
/**
 * URI is malformed/invalid
 * @param e UnknownHostException
 */
public void unknownHostExceptionThrown(UnknownHostException e);
/**
 * Problem reading variables, input, streams or strings
 * @param e IOException
 */
public void ioExceptionThrown(IOException e);
/**
 * Problem with the HTTP connection in the form of timeouts, too many retries, etc.
 * @param e org.apache.httpcomponent.HttpException, cast to generic Exception for convenience.
 */
public void httpExceptionThrown(Exception e);
/**
 * Problems with the socket, invalid SSL port or socket closed from service due to capacity problems.
 * @param e SocketException
 */
public void socketExceptionThrown(SocketException e);
/**
 * Input message is invalid or malformed.
 * @param e UnsupportedEncodingException
 */
public void unsupportedEncodingExceptionThrown(UnsupportedEncodingException e);
\end{lstlisting}

    The most important of these is the IOException method. This is the exception that will be thrown (As a straightforward IOException, or as a subclass of it called NoHttpResponseException) whenever the connection is closed without receiving the full reply from the server. In addition to the normal occurrences of this exception it will also happen whenever the server decides to cut a connection for priority reasons.

    It is worth noting that whenever an exception is thrown for a specific request, the response for that request is set to the name of the exception, so as to enable the client to find out which request has been terminated by an exception.

\subsection{Using the library}\label{userguideUsingLibrary}
    
    Once you’ve implemented the ExceptionHandler interface the client library can be easily constructed using:

\begin{lstlisting}[caption={Constructing the library}, label=userguideConstruction]
QoSClient client = New QoSClientImpl(String username, String userrole, String password, ExceptionHandler this);
\end{lstlisting}

Once you have a valid instance of QoSClient sending any data will be as simple as:

\begin{lstlisting}[caption={Sending data}, label=userguideSenddata]
ReceiveObject replyObject = client.send(String soap, URI destination);
String reply = replyObject.receive();
\end{lstlisting}

    The string you send has to be a valid soap message. The soap message has to contain a valid envelope with a body element, more on this in the caveats section.
    
    The ReceiveObject you get back from the send method is a blocking string, which means that calling receive on it will block execution until a reply is available.

\subsection{Using the library with listeners}\label{userguideUsingLibraryWithListeners}
    
    After constructing the library as shown above, you are able to add listeners to it using:
    
\begin{lstlisting}[caption={Add listener}, label=userguideAddlistener]
    client.addListener(DataListener listener);
\end{lstlisting}

    And remove them again using:
    
\begin{lstlisting}[caption={Remove listener}, label=userguideRemovelistener]
    client.removeListener(DataListener listener);
\end{lstlisting}

    The DataListener referenced here is the interface DataListener in the qos client library, it requires that you implement a single method:

\begin{lstlisting}[caption={The DataListener interface}, label=userguideDatalistener]
/**
 * Default receive method
 * @param recObj SOAP data
 */
public void newData(ReceiveObject recObj);
\end{lstlisting}

    Which will be called whenever a ReceiveObject gets some reply data.

\subsection{Change Credentials}\label{Change Credentials}
    
    If you for some reason wish to change the user credentials during execution, this can easily be done by calling:
    
\begin{lstlisting}[caption={Changing user credentials}, label=userguideCredentials]
    client.setCredentials(String username, String role, String password);
\end{lstlisting}

\subsection{Logging}\label{userguideLogging}

    The client library supports logging both to file and console, by default it only logs to console, and only warnings and above. To set whether to log to console you can call:

\begin{lstlisting}[caption={Turn logging to console on or off}, label=userguideTogglelogtoconsole]
    client.setLogToConsole(boolean on);
\end{lstlisting}

    To set logging to file:
    
\begin{lstlisting}[caption={Turn logging to file on or off}, label=userguideTogglelogtofile]
    client.setLogToFile(boolean on);
\end{lstlisting}

    To change the level of the logging between warning and above, or everything:

\begin{lstlisting}[caption={Switch between the two logging scopes}, label=userguideToggleloggingdetails]
    client.setFineLogging(boolean on);
\end{lstlisting}

\subsection{Caveats}\label{userguideCaveats}
\subsubsection{URI from client}
It is assumed by the client library that the port of the URI is a valid SSL port that is capable of communicating using TLS. It does however not do anything to validate the authority of the servers certificate and will accept any certificate.

\subsubsection{SOAP from client}
The client library requires that this is a valid SOAP envelope, with an element with the local name of “Body” one level inside the envelope. Note that all XML is case sensitive.

\subsubsection{User credentials}
User credentials are not checked for validity (beyond very basic sanity checking) until the library attempts to get a token.
For our implementation of the client server set, this doesn’t matter much since no credential validation is ever done, but it might matter if the client library is extended to interact with an actual identity server. If this is done, both the SAML communicator as well as the SAML parser should be reimplemented.

\subsubsection{Redundant token fetching}
If two or more requests are sent to the same service-set at the same time and the token has not already been fetched for that service-set there is a high probability that the token will be fetched over the network several times before being stored in the credential storage. Meaning you will waste network bandwidth getting the same information several times.

\subsection{Example code}\label{userguideExample}

A simple client that will send a soap containing “Hello World” to the service at “https://localhost:443/service/myService” and retry 4 times

\begin{lstlisting}[caption={A simple example client}, label=userguideExampleclient]
public class ExampleClient implements ExceptionHandler {
   final static int retry = 4;
   public static void main(String[] args) {
       URI destination = new URI("https://localhost:443/services/myService");
       QoSClient client = new QoSClient("John", "Anon", "DoeIsMe", this);
       String soap = "<?xml version=\"1.0\" ?><S:Envelope xmlns:S=\"http://schemas.xmlsoap.org/soap/envelope/\">" +
               "<S:Body>Hello World</S:Body></S:Envelope>";
       ReceiveObject recObj = null;
       int send = 0;
       do{
           recObj = client.send(soap, destination);
           send++;
       } while (recObj.receive().endsWith("Exception") && send<=retry);
       System.out.println(recObj.receive());
   }

   @Override
   public void unknownHostExceptionThrown(UnknownHostException e) {
       //Do nothing
   }
   @Override
   public void ioExceptionThrown(IOException e) {
       //Do nothing
   }
   @Override
   public void httpExceptionThrown(Exception e) {
       //Do nothing    
   }
   @Override
   public void socketExceptionThrown(SocketException e) {
       //Do nothing    
   }
   @Override
   public void unsupportedEncodingExceptionThrown(
           UnsupportedEncodingException e) {
       //Do nothing    
   }
}
\end{lstlisting}
