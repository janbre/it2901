\subsubsection{Weaknesses}\label{Testing:About:Weaknesses}
	There are some weaknesses connected with our testing suite and how we do the testing. Chief among them is the limited scope made necessary by limitations encountered in NS3. However there are other areas where the testing suite could be expanded which could be done without butting heads with NS3.
	
	One limitation that was self imposed is the fact that we only have one test network layout. This should have been expanded, but because of limited time at the end of the project we chose to focus more on the one test. With a more expanded network layout, which is quite feasible despite the problems encountered with NS3, one could add more clients and introduce several priorities to test how the ESB would behave. We theorize that the ESB should behave in the same way and we have created it in such a way to prioritize the highest priority messages no matter what the other messages does.
	
	Another limitation to the test setup is that we have no easy way to communicate between the LXCs. What this means is that we can not coordinate when to terminate the whole test. What this means is that we have to enforce a cutoff time which does skew some test. Especially bad is this when we run the test without our Throttle mediator. Because without our Throttle mediator even on the lower bandwidths no messages should be lost and the percentage of successful messages should be one. This will effect the test, but we decided to keep it this way because we mean the general trend on the results are still clear.
	
	The test client also has some problems. For one it does not try to retransmit any messages. This has a profound effect on the results regarding successful message percentage which will be quite different with and without our Throttle mediator. With our Throttle mediator we will perceive a lower percentage compared to without the mediator, but this is just a result of us dropping messages which retransmitting would to some degree correct. Another thing worth mentioning is that we have scheduled all the clients to start about the same time, there is a slight delay between them, but this scheduling means that on lower bandwidths there will be a distinct sending period where all the clients sends messages and there will be a distinct receiving period.
	
	For the lower bandwidth test there is also the fact that the client library uses HTTPS which we have observed in the lowest bandwidths do sometimes timeout because of the size of the handshakes which do somewhat interfere with the results. This should however be quite insignificant as the results sent back from the service is so large and would guaranteed timeout if the smaller handshake messages timeout.
	
	On the server side the biggest limitation is the static nature of the setup. We have tried to make the test so that we could test as much as possible, but there is one variable which we have not gotten to tweak. On the ESB we can configure the interval in which messages are taken out of the message store, but because of limited time to perform the tests we could not test this. For the results this means that the time taken for the lower priority clients will be a bit skewed, but again the trend should be clear.
