	\subsection{Test Cases}\label{Testing:Cases}

	As most of the tests below are quite alike the reasoning behind them are also quite like. The main difference between them are the “Timeout” which refer to the timeout the ESB uses. For more about the “Timeout” see \ref{Configuration of the ESB}.

	Since this project had somewhat of a research focus from the customers side we did not perform these test in order for us to validate our system. Instead we have run these tests to try and say something about the feasibility of the original question asked when we started. We will come back to this topic in the result section.
	
\begin{center}

\begin{tabular}{| p{4cm} | p{8cm} |}%\label{test:1}
    \hline
    ID & 1 \\
    \hline
    Description &  In this test what we are looking at is how our system behaves with a very low timeout, since we have full control over the message sizes sent in the test we know that this timeout will be too low on the lower bandwidths, but should perform much better on high bandwidths. \\
    \hline
    NS3 variables & Datarate: 1KBps,5KBps,10KBps,20KBps,40KBps \\
    \hline
    ESB variables & \textbf{Timeout: 500} \\
    \hline
    Automated & Yes \\
    \hline
    Expected Result & We expect to see that the ESB will preempt even higher priority messages in the lower bandwidth tests because of the low timeout, but on higher bandwidths the sending time of the all the messages should be lower than on the later tests. To put that in the same setting as our results, we expect the percentage of successfully received messages to be lower than in the tests below, but we expect the time to also be lower across the board.  \\
    \hline
\end{tabular}

\\ \ldots \\

\begin{tabular}{| p{4cm} | p{8cm} |}%\label{test:2}
    \hline
    ID & 2 \\
    \hline
    Description & In this test we have increased the timeout substantially, we expect to see some improvements on 10KBps and still retain some of the benefits of a lower timeout on 20- and 40KBps \\
    \hline
    NS3 variables & Datarate: 1KBps,5KBps,10KBps,20KBps,40KBps \\
    \hline
    ESB variables & \textbf{Timeout: 1000} \\
    \hline
    Automated & Yes \\
    \hline
    Expected Result & We expect to have a higher percentage of completed messages on 10KBps than with a timeout of 500 and we expect the results on 1-, 20- and 40KBps to be relatively unchanged. \\
    \hline
\end{tabular}

\\ \ldots \\

\begin{tabular}{| p{4cm} | p{8cm} |}%\label{test:3}
    \hline
    ID & 3 \\
    \hline
    Description & Again we have increased the timeout and expect to see some improvements on percentage, but we the total time taken should start to drop on higher bandwidths.  \\
    \hline
    NS3 variables & Datarate: 1KBps,5KBps,10KBps,20KBps,40KBps \\
    \hline
    ESB variables & \textbf{Timeout: 2000} \\
    \hline
    Automated & Yes \\
    \hline
    Expected Result & We expect all the messages on 20 and 40KBps to arrive, we expect that on 10KBps more messages should arrive, but not all. The time taken should again increase. \\
    \hline
\end{tabular}

\\ \ldots \\

\begin{tabular}{| p{4cm} | p{8cm} |}%\label{test:4}
    \hline
    ID & 4 \\
    \hline
    Description & In this test we want to see how the ESB copes with a much larger timeout.  \\
    \hline
    NS3 variables & Datarate: 1KBps,5KBps,10KBps,20KBps,40KBps \\
    \hline
    ESB variables & \textbf{Timeout: 5000} \\
    \hline
    Automated & Yes \\
    \hline
    Expected Result & We expect that 10-, 20- and 40KBps should be enough to get most of the messages for the high priority client through, the time taken should again increase and this should be noticeable on 40KBps compared to Test 1.  \\
    \hline
    \end{tabular}
    \\ \ldots \\
    \begin{tabular}{| p{4cm} | p{8cm} |}%\label{test:5}
    \hline
    ID & 5 \\
    \hline
    Description & In this test we have gone all out. The timeout is massivly increased to see how the ESB behaves on the lowest bandwidths, 1- and 5KBps respectively.  \\
    \hline
    NS3 variables & Datarate: 1KBps,5KBps,10KBps,20KBps,40KBps \\
    \hline
    ESB variables & \textbf{Timeout: 100 000} \\
    \hline
    Automated & Yes \\
    \hline
    Expected Result & We expect the same percentage on 10-, 20- and 40KBps as Test 4. What we want to see is that on 5KBps the percentage is increased quite substantially compared to the previous tests. \\
    \hline
\end{tabular}

\\ \ldots \\

\end{center}
