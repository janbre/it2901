\section{Project Introduction}\label{Project Introduction}
    This is the final report documenting our progress so far in the course IT2901 - Informatics Project II. It will give a description of the problem we were presented with, how we have planned the work for the weeks ahead, how we have organized our group and what development methodology we have chosen etc. We will also give a brief discussion about why we have made the decisions we have.
    
    \subsection{Project Background}\label{Project Background}
    \begin{quotation}
    \em Essential to Network Based Defense (NBD) is the concept of end-to-end QoS, which in turn requires employing cross-layer QoS signaling. This means that QoS must be considered at all layers of the OSI model, and that QoS information must traverse these layers\ldots -Motivation\footnote{Please see (\ref{File Attachments}) - Quality of Service (QoS) support for Web services in military networks}
    \end{quotation}
    This was the introduction we got for our motivation to work on this project. In many ways it illustrates the background of the project and why the customer wants us to work on it. As the customer is working with wireless networks with very low bandwidth they need to be able to control the flow of messages. The reason for these strict requirements is the command hierarchy and the risks involved which means that some messages in the network are more important than other. To be able to separate those messages there needs to be a collaboration between all the applications and libraries used. Currently there is no or little support for this cooperation on the application level which is what we are tasked to do. For us this assignment will be a challenge not just because it is some what uncharted waters, but also because of the strict requirement to prioritize.

    \subsection{Customer}\label{Customer}
    Our customer are two senior researchers at the Norwegian Defence Research Establishment(FFI)\footnote{\href{http://www.ffi.no/no/Sider/default.aspx}{FFI}} working in the SOA division. They had this to say about FFI.
    \begin{quotation}
    \em The Norwegian Defence Research Establishment (FFI) is the prime institution responsible for defence-related research in Norway. The establishment is the chief adviser on defence-related science and technology to the Ministry of Defence and the Norwegian Armed Forces’ military organization. FFI addresses a broad spectrum of research topics ranging from the assistance of operational units to the support of national security policy via defence planning and technology studies. FFI is located at Kjeller, 2 km north of Lillestrøm and 25 km from Oslo. FFI also has a research unit in Horten focusing on maritime research.
    \end{quotation}
    
    \subsection{Course}\label{Course}
    \begin{quotation}
    \em In this course, students will work in groups to carry out a software project. The department will present a list of available projects. Students are required to work on their project and to attend common activities and supervision meetings. The results from each phase must be clearly documented in the mid-term and final report\ldots\footnote{Description of the course on the homepage of \href{http://www.ntnu.edu/studies/courses/IT2901}{NTNU} fetched 29.03.12}
    \end{quotation}
    
    \subsection{Students}\label{Students}
    
    \begin{Student}
    \indent \indent \textit{Bremnes, Jan A. S. } \\
        texts here
        \\
    \end{Student}
    
    \begin{Student}
    \textit{Johannesen, Stig Tore} \\
        texts here
        \\
    \end{Student}
    
    \begin{Student}
    \textit{Kirø, Magnus L.} \\
        This document is structured in a fashion to show you, the reader, our process through this project. It should give you a view into each part of the project from the early weeks when we were struggling to understand the initial design right up until our final moments with acceptance testing.
        \\
    \end{Student}
    
    \begin{Student}
    \textit{Nordmoen, Jørgen H.} \\
        texts here
        \\
    \end{Student}
    
    \begin{Student}
    \textit{Støvneng, Ola Martin T. } \\
        texts here
        \\
    \end{Student}
    
    \begin{Student}
    \textit{Tørresen, Håvard } \\
        texts here
        \\
    \end{Student}
    
    \subsection{Document Structure}\label{Document Structure}
    This document is structured in a fashion to show you, the reader, our process through this project. It should give you a view into each part of the project from the early weeks when we were struggling to understand the initial design right up until our final moments with acceptance testing. 
    
    We will start with \emph{Prestudy} which details our initial thoughts around the project. It includes the architecture we envisioned for the client and the server early on in the project and it will give you some insight into what we initially thought and will serve as a good comparison with our end result.
    
    Next is the chapters about \emph{Project Management} and \emph{Development Methodology}, these two chapters combined should give an impression of our thoughts and plans for collaborations to reach our end goals. It should also give you an idea about how all of our appended documents, such as Activity Plans, have played a role during the course of the project. After reading theses sections, the team structure and the distribution of responsibilities in the group, should be explained.
    
    The following chapters outline our requirements and the design for our solution. \emph{Task Description and Requirements} details the task we were given by the customer, and the high level requirements that we, together with the customer, agreed upon. The chapter about \emph{Design} will focus on the design of our two pieces of software in some detail and should give you a good idea about the overall architecture of the system. After that follows \emph{Implementation} which should give the necessary details about our code and how the system works deeper down.
    
    The next chapter outlines our \emph{Testing}. It shows how we have chosen to do unit testing and all the way up to system testing. It will also give you the knowledge to reproduce our testing framework. This ability to reproduce our whole testing setup should also give you the confidence that we have in our results. \emph{Results} will give you our interpretation of the results that we have gotten from the system test and should also shed light on the initial problem we set out to investigate.
    
    Finally we will wrap up with \emph{Conclusion} and \emph{Project Evaluation} which together will wrap up the report. This will look at the future work for this research and give you our final thoughts about the course, project and process.
    
    
