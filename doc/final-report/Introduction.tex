\section{Project Introduction}\label{Project Introduction}
    This is the final report documenting our progress so far in the course IT2901 - Informatikk Prosjektarbeid II. It will give a description of the problem we were presented with, how we have planned the work for the weeks ahead, how we have organized our group and what development methodology we have chosen etc. We will also give a brief discussion about why we have made the decisions we have.
    
    \subsection{Project Background}\label{Project Background}
    \begin{quotation}
    \em Essential to Network Based Defense (NBD) is the concept of end-to-end QoS, which in turn requires employing cross-layer QoS signaling. This means that QoS must be considered at all layers of the OSI model, and that QoS information must traverse these layers\ldots -Motivation\footnote{Please see (\ref{File Attachments}) - Quality of Service (QoS) support for Web services in military networks}
    \end{quotation}
    This was the introduction we got for our motivation to work on this project. In many ways it illustrates the background of the project and why the customer wants us to work on it. As the customer is working with wireless networks with very low bandwidth they need to be able to control the flow of messages. The reason for these strict requirements is the command hierarchy and the risks involved which means that some messages in the network are more important than other. To be able to separate those messages there need to be a cooperation between all the applications and libraries used, and currently there is no or little support for this cooperation on the application level which is what we are tasked to do. For us this assignment will be a challenge not just because it is some what uncharted waters, but also because of the strict requirement to prioritize.

    \subsection{Customer}\label{Customer}
    Our customer is two researchers at the Norwegian Defence Research Establishment(FFI)\footnote{\href{http://www.ffi.no/no/Sider/default.aspx}{FFI}} working in the SOA division.
