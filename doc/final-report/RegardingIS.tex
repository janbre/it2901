Our original design called for the implementation of the WSO2 Identity Server \LINK I FOTNOTE ELLER REFERANSE TIL BIB, but our final product does not include it. We ran into some problems while trying to set it up, and after discussing it with the customer, they agreed that we could drop it. It was, after all, not a functional requirement from their side, though it would be preferable to have it included. 

We started our research of the IS too late, only two weeks before our prototype demonstration, and we only had one group member working on it. After seeing how well documented and fairly easy to use the WSO2 ESB was, we figured the IS would be equal. We figured wrong. 

The WSO2 Identity Server product page at \url{www.wso2.org} is severly lacking in documentation. The user guide and administration manual contains barely no information about how to set it up, configure it and use it. They provide links to some blog posts that employees had written back in 2009, and even though they contained some useful information, and sometimes provided example configuration files and client code, they did not state which versions of the different products they were basing their examples on. This could have been the cause of some of the problems we had with the configuration files, as the client code would not accept them, stating that they were not of the correct format, but the IS and the ESB accepted them. If we changed the configuration files so that the client would accept them (the only adjustment needed was to change the name of a tag in the WS-security policy XML-file, from <sp:Policy> to <wsp:policy>) then the IS and ESB would not accept them. As it was just a minor adjustment, changing the name of a tag, the rest of the policy was identical, we don't believe this caused any problems, but it is an example of how frustrating it could be trying to use the code provided, as it was poorly commented and made some assumptions about previous knowledge of how the mentioned Apache products worked and that you already had the .jar files for these products on your classpath. 


We used this blog post LINK TIL BLOGG to try and set up communication between the IS and ESB. It was not entirely similar to our use case, as it uses X509 signing and encryption with HTTP transport, instead of using HTTPS and let the transport layer take care of security. After much trial and error, a lot of exceptions and googling for solutions, we managed to get the IS to issue security tokens and send them to the ESB, but the ESB failed in decrypting and verifying them. We were unable to figure exactly why it failed, as the error message we got was that the ESB could not find the public key of the Identity Server, but using the Java keytool, we could verify that the key was present in the ESB key store. After spending quite a few hours trying different solutions, exporting the IS public key and importing it to the ESB key store under a new alias, importing the ESB public key into the IS key store and etc. we gave up, as this specific use case was not the one we were after, and we had at least succeeded in getting the ESB and IS to talk to each other. 

Next, we tried configuring the IS to issue username tokens and send them over HTTPS, which would remove the need for endpoint encryption and was after all the use case we had in mind. We could find no specific examples for this scenario, so we tried creating our own security policy file, and adjust the settings in the ESB and IS. We managed to get the IS to create username tokens, but nothing more, as it crashed when trying to send it to the ESB, stating that the SOAP header did not include a security element. Monitoring the SOAP messages we could see that the header infact did contain this element, so we don't know why the IS couldn't find it. 



Had any of us had some experience with WS-security from before, and been familiar with the WS-security policy language, and the Axis2, Axiom, Rampart and Tomcat from Apache
