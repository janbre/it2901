% the flat structure worked out well
% we had no conflicts in the group. 
% the task distribution that automatically happened 
% comment on how this worked: some people had the unofficial responsibility for some tasks. Jan - IS, Jørgen - NS3, Ola - rooms, Magnus - Report, Stig - client. 
% evaluate the work schedule - mon-thur/10-16.
The problem we had with the roles in the group, is that we did not manage to switch betweeen them during the project. This resulted in some instances where we could not proceed with something, because the one who knew the most about it was ill. This was something that we were quite wary of in the beginning of the project, but we did not follow it up with the same care. The reason behind this is partly that it was usually easier to simply ask the person in charge of that specific part, to explain what you did not understand. Another reason was that we worked together every day during the project, which probably gave us the false confidence that we would not need to share the information. The few early roadblocks that we did encounter during the start up, where we could have done something about it, were probably so small that we just carried on without thinking about it.

We feel that the flat structure we chose worked out quite nice. When we came to a big decision that one of us felt uneasy about taking full responsibility for, we talked about it in the group, and came to a mutual decision. There are many reasons why this worked in our group, but a key enabler was the fact that we worked together Monday to Thursday. This close proximity made such decisions easy to approach, and agreement could quickly be made. If we had not had this work schedule, this sort of management might not have worked out the way it did.

We mentioned roles above, but we would also like to mention a bit about how that worked on a daily basis. Because we grew into certain roles as the project progressed, we also got some responsibilities. For instance, oneof us quickly became the main contact between the group and the supervisor, but having only one person handling that communication did not hinder use. In fact, the communication became quicker, because the supervisor could relate to one person and not everyone, and the fact that when someone in the group wanted to ask the supervisor about something, they could contact Magnus.

Luckily, we had very few conflicts in the group. There was some miscommunication which led to some disagreements, but there where very few of those. Among the few things that came up during this 15 week long project, was a miscommunication about when we were going to start working again after the Easter vacation. This little incident only lead to some stressful days, and the misunderstanding was cleared away.

As mentioned above, we tried to work together from Monday to Thursday from 10:00 to 16:00. This worked rather well during the whole project, except for some hiccups regarding the "early" start. We had some weeks where not all of us managed to get up in time, but after a meeting within the group we cleared away any bad air, and did not really experience any such hiccups after that.
