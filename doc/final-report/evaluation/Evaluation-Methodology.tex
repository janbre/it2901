% Scrum 
% Agile
% Did the metodology work for us?
% Could we have adapted better to our methodology?
In this project we went against the current when it comes to modern software development methodology. Instead of going with the darling of the development world, SCRUM, we chose to go back and pick something that most would not. The waterfall model might not be the best fit for everyone, but for this project we think we got it right.

The first thing we knew was that most of the technologies was unfamiliar for everyone, even the customer, this meant that we had to invest a lot of time getting to know the different frameworks. Another thing was the research focus of the project, which demanded a certain investment into planning. All this lead us to the waterfall model, explained in section ~\ref{Software project life cycle}, which worked out quite well. We had a large planning phase, which included some head-scratching moments, but for the most part, we got through it. The implementation phase had some more problems, mostly with regards to time estimates and external dependencies. Had we gone for a more agile methodology, we might not have ended up with such a thorough design.

Although the previous paragraph does mention that we chose the waterfall model, we did not completely forget the agile world. Most of the implementation went according to a more agile development methodology, where we had weekly sprints, tried to have code reviews and used unit testing. We feel that this mixing of methodologies is some of the things we excelled at the most during this project. The waterfall model helped us with the planning and design of an unknown entity, and the agile implementation lead to cleaner and better code quality.
