Since the project contained such a large design phase it is only natural that we elaborate on what that phase entailed.

We started the design phase with a vague idea about what we were going to create. With the help of the customer we got a little further and soon we were going strong. Because of the nature of the project we did not have a clear starting point. So in the beginning we followed every hint that the customer gave us and tried to use every framework that the customer said could help. This gave us a lot to do because we had to get familiar with a lot of frameworks and asses their usefulness for the project. We then came up with the design which you can see in the prestudy section \ref{Server side Architecture} and \ref{Client side Architecture} which we used as a spring board into the actual design.

After this initial design we first checked with the customer that the design was something like what they had in mind and then went forward with our design to come up with a more final version. Since we had gotten a little more confidence in the design we started putting more effort into the different frameworks that we wanted to use and developing a better image of what we could create.

The design phase went quite well, and the only real problem we had were problems related to unfamiliarity with the frameworks we decided to use. These problems subsided over time as our familiarity grew.

Most of what we decided during the design phase were possible, but some things were out of our reach. Among other things the design of the testing suite had to be scaled back quite a bit, but everything that we had to alter during implementation were thoroughly discussed with the customer so that they had a saying in every alteration.

The design phase overall were quite helpful and since we initially had no idea about how the final design would look the large design phase really helped. We learned a lot during those weeks and we have a new understanding of what is required during design for a large scale project such as this.
