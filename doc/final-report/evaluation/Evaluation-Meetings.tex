% What worked? 
% What did not?
% Supervisor meetings was kind of meaningless in the beginning. 
% FFI meetings did occure frequently and smoothly. 
% FFI meetings mostly gave us a good relationship with the customer. The actual information shared and the topics discussed would have been equally good or better discussed by email. 
When it came to meetings, we did a decent job. As mentioned before we had biweekly meetings with the supervisor, and weekly meetings with the customer over Skype. The meetings went smoothly, but we did not prepare well for most meetings which reflected in the length and content of the meetings. This often lead to having to take followup questions over email, which for the most part went well, but on some occasions we could definitely have benefited from better meeting preparation.

The distance to the customer never posed any real problems to the meetings, we never rescheduled any meetings, and even when the customer came to Trondheim, all went well. The only thing which could have caused a problem regarding distance was the problem of impromptu meetings, but the customer never said no to another meeting later in the week if we needed it, so even this did not cause a problem. 

The meetings with the supervisor also went well. We had no problems scheduling meetings and the supervisor was very flexible when it came to the meetings. The meetings were often informal and the tone between us and the supervisor was good.

Other than the lack of preparation before the meetings, the meeting part of the project was good. If there is anything to draw from our experience around meetings, it is that proper preparation is worth doing.
