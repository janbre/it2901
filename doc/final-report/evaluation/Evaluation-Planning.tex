In the first couple of weeks we spent a lot of time trying to understand the task and what we were supposed to do. We didn't plan the process very well. This was probably because we didn't know exactly what to plan for. What we did instead was to make daily agendas of what we should research.

After the first few weeks, we had managed to get a decent idea of what needed to be done, and so the more detailed planning began. We started by making a Gantt chart (ref:Appendix~\ref{attachment:file:Gantt} - gantt.html) with the different work packages we needed to be complete by what dates. At this point we should have planned better and updated our gantt chart along the way. In the end we did keep most of the deadlines we set: we started implementation when intended, and we only went a week past the intended deadline for implementation (excluding some important bugfixes). 

The biggest miss in the initial plan was probably that we intended to have the final report ready by April 16th, so that we could get detailed feedback on it before the final delivery. This ended up not happening because implementation required more resources than anticipated, and we prioritized the product before the report.

We also started making weekly Activity Plans (ref:~\ref{Attachments:Activity Plans}), instead of the daily agendas. The first two weeks of activity plans were very badly done. But after that we made more detailed and more accurate plans.

Work breakdown structures were also made. They started out somewhat inaccurate, but by the end of the design phase they had become more correct and informative. 

Time estimation on the different tasks in the activity plans were very difficult. Some tasks ended up taking more than twice or even three times as much time as anticipated, while other tasks ended up taking less than half of what was planned. But most of the time we were within 20\% of the anticipated time, which we believe is pretty good. The activity plans proved to be the most useful planning tool for us. This despite the inacurracy in time estimation. The best effect of the activity plans was focus. We would look at the plan and see what the next task was. This kept us on course. 

Towards the end of the project the routine of creating activity plans faltered. This was due to bug hunting and report feedback. We tried to create activity plans but they ended ut being a sort of checklists for report improvements. And while the bughunting tok priority towards the prototype meeting, updating the activity plans was forgotten.
    
The bughunting and code improvements after our prototype presentation has not been planned or tracked in any way. This is a very bad practize and we look back on it as a point where we have massive room for improvement.

