In the first couple of weeks we spent a lot of time trying to understand the task and what needed to be done. We did not plan this process very well. This was probably because we did not know exactly what to plan for. What we did do was make daily agendas of what we should research.

After the first few weeks, we had managed to get a decent idea of what needed to be done, and so the more detailed planning began. We started by making a Gannt diagram (ref:Appendix \ref{attachment:file:Gantt} - gantt.html) with the different work packages we needed done by what dates. We should probably have planned in more detail at his point, and updated the gantt diagram when we learned more. In the end we did keep most of the deadlines we set: we started implementation when intended, and we only went a week past the intended deadline for implementation (excluding some important bugfixes). 

The biggest miss in the initial plan was probably that we intended to have the final report ready by April 16th, so that we could get detailed feedback on it before the final delivery. This ended up not happening because implementation required more resources than anticipated, and we prioritized the product before the report. We did not think of this as a problem, as we knew that we would get good feedback from our customer, supervisor and possibly some other people.

<<<<<<< HEAD
We also started making weekly Activity Plans (ref:\ref{}), instead of the daily agendas. The first two weeks of activity plans were very badly done, and not included in the report. But after that we made more detailed and more accurate plans.

Work breakdown structures were also made. They started out somewhat inaccurate, but by the end of the design phase they had become more correct and informative. 
=======
Work breakdown structures were also made, they started out somewhat inaccurate, but by the end of the design phase they had become more accurate and more informative. 
>>>>>>> 006b119841ced13445af8ddda08713ee6fb54822

We also started making weekly Activity Plans (ref:\ref{}) instead of the daily agendas. The first two weeks of activity plans were rather badly done and not included in the report. But after that we made more detailed and more accurate plans.

Time estimation on the different tasks in the activity plans were, as always, very difficult. Some tasks ended up taking more than twice or even three times as much time as anticipated, while other tasks ended up taking less than half of what was planned. But most of the time we were within 30\% of the anticipated time, which we believe is pretty good. These activity plans proved to be the most important planning tools we used, even though they were a bit off from time to time they really helped us focus on what needed to be done from week to week.

Towards the end of the project, with bug hunting and finalizing the report, we ended up not making activity plans. We did not make plans for bug hunting because bugs are generally hard to anticipate. But the report writing could probably have been planned better.
