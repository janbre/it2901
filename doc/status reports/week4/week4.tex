\documentclass[titlepage]{article}
\usepackage{amsmath}
\usepackage[utf8]{inputenc}

\title{Project: FFI - QoS \\ Summary status report \\ Week 4}
\author{Nordmoen, Jørgen \and Tørresen, Håvard \and Bremnes, Jan A. S.  \and Kirø, Magnus  \and Støvneng, Ola Martin  \and Johanessen, Stig Tore}
\begin{document}

    \maketitle
    \pagenumbering{arabic}
    \newpage
      
    \section*{Introduction}
        Our task is to build a prototype Quality of Service (QoS) layer to web services for use in military tactical networks. These networks tend to have severely limited bandwidth, and our QoS-layer must prioritise between the different messages$/$packages, of varying importance, that clients and services want to send. Our middleware will have to recognize the role of clients, and together with the service they are trying to communicate with, decide what sort of quality the messages should get.
The prototype should be able to run on regular off-the-shelf laptops. Support for more mobile and resource-limited hardware is not expected nor required, as full Java support is needed.

    \section*{Progress summary}
        Work on the project started Wednesday last week, when we received notification of which project we were assigned. We used Wednesday and Thursday to decide how we should organize the work, which tools to use, and set up a weekly schedule.
We have decided that from Monday to Thursday each week, the whole group will work together from 10:00 - 16:00, minus lectures. Every Thursday, we'll write a weekly status report which will be submitted to both our supervisor and the client. Every Friday we will have a meeting with our supervisor, and meetings with the client will be held over Skype every Tuesday.
\\\\
        The prototype will be written in Java, each group member is free to use his preferred IDE, and standard Java Code Conventions will be followed. 

\\To manage our source code, we will use Git via Github, where our project will be available as open source. The client has agreed to this. We will at all times maintain two branches of the source; a Master branch and a Testing branch. All development will be done against Testing. Every other week, we'll do a full test of the Testing branch, and if it's declared healthy, it will be merged into the Master branch as soon as possible. After merging, the Testing branch will again be open for development.

\\\\
        We will apply test driven development, so we will write tests in JUnit before writing the actual code. The client has stated that a thoroughly tested and stable prototype with limited functionality is preferred over a feature-rich, unstable prototype, so we will focus on getting the basic functionality working properly.
\\\\
        Google docs will be used for collaboration on reports, TODOs, group meeting summaries etc. All reports and documentation to be submitted for evaluation, will be submitted as PDF.
We had a physical meeting with the client on Tuesday 24th, where we discussed the assignment to get more detailed information about what is expected of us. We were presented with different technologies and web standards that should be used and implemented, examples are SAML, XACML, WSO2 and Glassfish. We have started researching and experimenting with these technologies in order to see what tools they support us with. 
\\The meeting, together with two reports that the client has sent us regarding previous work done on the subject, gave us a clear idea of our task.
\\\\
        Work have also been started on the preliminary report, and we plan on having our next physical meeting with the client on Feburary 8th.


    \section*{Completed tasks}
        We have defined a weekly schedule, and agreed on the tools to be used
\\Created a repository on Github, created templates for reports etc
\\Had our first meeting with the client
\\Created a technical glossary for the project, which will be shared with the client so that we have a clear definition of the technical terms and expressions

    \section*{New tasks}
        Technology research (Glassfish, WSO2)
\\Get access to a virtual machine from IDI, and set it up with the relevant services
\\Make an outline of the planned architecture

    \section*{Planned work for next period}
        We have not set up a detailed activity plan yet, as our focus the next few days are to learn more about WSO2 ESB and Glassfish, and to decide which parts we can/should use, and how to integrate them. The preliminary report must be finished by the end of next week and we hope we’ll have an outline of the planned architecture of our prototype. We will request access to a virtual machine, and set it up according to our needs.

\end{document}

