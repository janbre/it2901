\section{Techical Glossary}
\begin{description}\label{glossary}

%B
\item[Bandwidth]\label{glossary:bandwidth} \hfill\\
Available or consumed data communication resources \\ \url{https://secure.wikimedia.org/wikipedia/en/wiki/Bandwidth_(computing)}

\item[Broker]\label{glossary:broker} \hfill\\
Our middleware layer works as a QoS broker for services and clients \\
Broker as referred to by the report given to us from FFI: a centralized ‘server’ of sorts which gathered Bandwidth data from tactical routers.

%C
\item[COTS]\label{glossary:cots} \hfill\\
Commercially available Off-The-Shelf often used to talk about services which the customer wants to use server side \\ \url{https://secure.wikimedia.org/wikipedia/en/wiki/Commercial_off-the-shelf}

\item[Credentials]\label{glossary:credentials} \hfill\\
User-supplied credentials in the form of a username, password, role tripple.

%D
\item[DiffServ]\label{glossary:diffserv} \hfill\\
Differentiated services, a field in the IPv4 header \\ \url{http://www.networksorcery.com/enp/rfc/rfc2474.txt}

%G
\item[Git]\label{glossary:git} \hfill\\
A free and open source, distributed version control system \\ \url{http://www.git-scm.com}

\item[Github]\label{glossary:github} \hfill\\
a web-based hosting service for software development projects that use the Git version control system \\ \url{http://www.github.com}

\item[GlassFish]\label{glossary:glassfish} \hfill\\
Application server written in Java \\ \url{http://glassfish.java.net/}

%I
\item[IDE]\label{glossary:ide} \hfill\\
Integrated Development Environment

\item[Identity Server]\label{glossary:identity server} \hfill\\
\url{http://wso2.com/products/identity-server/}

%J
\item[Java Coding Conventions]\label{glossary:java coding conventions} \hfill\\
\url{http://www.oracle.com/technetwork/java/codeconv-138413.html}

\item[JUnit]\label{glossary:junit} \hfill\\
A testing framework for the Java programming language \\ \url{http://junit.org/}

%L
\item[LaTeX]\label{glossary:latex} \hfill\\
A document preparation system \\ \url{http://www.latex-project.org/}

%M
\item[Mediator]\label{glossary:mediator} \hfill\\
A component in WSO2 ESB which can be used to work on incoming or outgoing messages that passes through the ESB \\ \url{http://synapse.apache.org/Synapse_QuickStart.html}

\item[Message]\label{glossary:message} \hfill\\
SOAP message  \\ \url{https://secure.wikimedia.org/wikipedia/en/wiki/SOAP#Message_format}

\item[Message Context]\label{glossary:message context} \hfill\\
Component in the ESB, contains the message, as well as all information about it, including network sockets. \\ \url{http://synapse.apache.org/apidocs/org/apache/synapse/MessageContext.htm}l

\item[Middleware]\label{glossary:middleware} \hfill\\
In most reports middleware will refer to the program we are making. Other distinctions should be made explicitly in the text.

\item[MS]\label{glossary:ms} \hfill\\
Monitoring Service, a service that provides bandwidth monitoring, running on the same server as the Tactical Router.

%N
\item[NS3]\label{glossary:ns3} \hfill\\
A network simulator \\ \url{http://www.nsnam.org/}

%O
\item[OpenSAML]\label{glossary:opensaml} \hfill\\
A set of open source C++ & Java libraries to support developers working with SAML. \\ \url{https://wiki.shibboleth.net/confluence/display/OpenSAML/Home/}

%P
\item[Packet]\label{glossary:packet} \hfill\\
IP packet  \\ \url{https://secure.wikimedia.org/wikipedia/en/wiki/Transmission_Control_Protocol#TCP_segment_structure}

%Q
\item[QoS]\label{glossary:qos} \hfill\\
Quality of Service refers to several related aspects of telephony and computer networks that allow the transport of traffic with special requirements

%S
\item[SAML]\label{glossary:saml} \hfill\\
Security Assertion Markup Language  \\ \url{https://secure.wikimedia.org/wikipedia/en/wiki/SAML}

\item[SOAP]\label{glossary:soap} \hfill\\
A lightweight protocol intended for exchanging structured information in the implementaion of web services in computer networks \\ \url{http://www.w3.org/TR/soap12-part1/#intro}

%T
\item[Tactical Router]\label{glossary:tactical router} \hfill\\
Router used in military networks, Multi-Topology Router(MTR)

\item[Token]\label{glossary:token} \hfill\\
 A SAML token from some form of identity server, possibly with additional meta data.

\item[TOS]\label{glossary:tos] \hfill\\
Type of service, a field in the IPv4 header, now obsolete and replaced by diffserv

%W
\item[Web Service]\label{glossary:web service} \hfill\\
A software system designed to support interoperable machine-to-machine interaction over a network \\ \url{http://www.w3.org/TR/2004/NOTE-ws-gloss-20040211/#soapmessage}

\item[WS-Security]\label{glossary:ws-security} \hfill\\
An extension to SOAP to apply security to web services

\item[WSO2 ESB]\label{glossary:wso2 esb} \hfill\\
An Enterprise Service Bus built on top of Apache Synapse \\ \url{http://wso2.com/products/enterprise-service-bus/}

%X
\item[XACML]\label{glossary:xacml} \hfill\\
eXtensible Access Control Markup Language  \\ \url{https://secure.wikimedia.org/wikipedia/en/wiki/Xacml}

\end{description}
